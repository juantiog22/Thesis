\chapter{Conclusiones}

Para concluir nuestro proyecto, vamos a analizar si los requisitos propuestos han sido llevados a cabo. Finalmente expondremos algunos aspectos a mejorar como trabajo futuro y una opinión personal evaluando mi experiencia tras el desarrollo del proyecto. 

\begin{itemize}
    \item Se ha conseguido implementar un chatbot que recoja datos relacionados con el bienestar de las personas.
    \item Se ha conseguido diseñar una arquitectura y una base de datos que cumple con lo que necesitábamos. Tanto la interfaz web, como la base de datos y el bot en telegram se encuentran mutuamente conectados. La base de datos contiene la información  de usuarios, preguntas, bloques, mensajes, respuestas y todo almacenado de forma clara. 
    \item La interfaz web responde adecuadamente a lo que estábamos buscando. Una aplicación sencilla e intuitiva para poder interactuar con el bot y las respuestas recogidas.
    \item El bot es capaz de tener unaz conversación con el usuario de forma escueta para satisfacer la interacción. Además de realizar el cuestionario cuando sea oportuno.
    \item El almacenamiento de las respuestas en la base de datos se realiza de forma automática y correcta, con la funcionalidad añadida de poder exportarlas.
\end{itemize}

El desarrollo del proyecto se ha alargado algo más de lo planteado, debido a ciertos cambios en el diseño que conforme se ha ido implementado han ido surgiendo, también sumado al desconocimiento de algunas tecnologías usadas. Pero tras tener todo esto en cuenta, el resultado ha sido satisfactorio ya que todo los requisitos planteados han sido cubiertos. 



 \section{Aspectos a mejorar}

La funcionalidad de nuestro proyecto puede ser mejorada de diversas formas. En el caso del bot sobre todo, se puede mejorar el tema de la interacción con el usuario usando sistemas de procesamiento de lenguajes más avanzados. También se podría exportar a otras plataformas y que pueda llegar a más personas. En el caso de la aplicación web un punto potencial de mejora sería la creación de una API que nos permita la interacción con la base de datos de una forma más eficiente. Además de algunos buscadores y filtros por parámetros dentro de las vistas de tablas. 


 \section{Valoración personal}

 Ha sido mi mayor reto afrontado en el transcurso de mi etapa como estudiante en la Universidad de Granada al comprometerme para su desarrollo en un entorno de trabajo real. Además sumado a lenguajes, técnicas y áreas donde no tenía especial conocimiento, pero el interés por mi parte a todo tipo de competencia han marcado una etapa de aprendizaje de forma individual bastante enriquecedora. En esta línea, he intentado utilizar siempre cosas útiles y empleadas por programadores y empresas en la actualidad, por lo que todo lo adquirido me ayudará en el desarrollo de futuros proyectos y trayectoria profesional.

 Tras haber realizado este proyecto me he dado cuenta de cómo las tecnologías pueden enriquecer y facilitar nuestras vidas cotidianas. Observamos que la tecnología en sí misma no es un fin, sino un medio para lograr el florecimiento humano. A medida que nos adentramos cada vez más en un mundo más tecnológico, es fundamental reconocer que estas herramientas son manifestaciones de la capacidad humana para imaginar, crear y transformar su realidad. 

\begin{quote}
    \textit{"La tecnología no es nada. Lo importante es que tengas fe en la gente, que sean básicamente buenas e inteligentes, y si les das herramientas, harán cosas maravillosas con ellas."} \\ 
    -- Steve Jobs.
\end{quote}
