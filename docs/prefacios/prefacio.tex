\chapter*{}
\thispagestyle{empty}


\begin{center}
{\large\bfseries POSTCOVID-AI Chatbot: facilitating data collection of people level human behavior }\\
\end{center}
\begin{center}
Juan del Río Gómez\\
\end{center}
\vspace{0.7cm}
\noindent{\textbf{Palabras clave}: chatbot, bienestar, cuestionarios, aplicación web}\\

\vspace{0.7cm}
\noindent{\textbf{Resumen}}\\

Este proyecto se centra en el desarrollo de un agente conversacional que permita la recolección de datos relacionados con el bienestar de las personas. Forma parte del proyecto POSTCOVID-AI como solución escalable a su método de recogida de datos mediante uso de otras tecnologías. El objetivo es facilitar esta recogida de datos para su posterior análisis a través de una herramienta que sea fácil de utilizar y atractiva a todo usuario.\vspace{0.3cm}

El chatbot guia al usuario a través de preguntas específicas sobre su estado de ánimo, actividades diarias, hábitos de sueño y alimentación entre otros aspectos. Además de recoger datos, el bot esta programado mediante un conjunto de instrucciones establecidas que permiten a los usuarios interactuar con él.\vspace{0.3cm}

Gracias a estas pautas, la persona puede tener una conversación con el bot y a su vez este le irá haciendo una serie de preguntas previamente definidas. Todo esto se puede configurar a través de una aplicación web orientativa a la que se tendrá acceso para poder modificar las preguntas a realizar, agruparlas, añadir mensajes al bot, planificar cuestionarios y consultar los datos en tiempo real.


\cleardoublepage
\thispagestyle{empty}

\begin{center}
{\large\bfseries POSTCOVID-AI Chatbot: facilitating data collection of people level human behavior }\\
\end{center}
\begin{center}
Juan del Río Gómez\\
\end{center}
\vspace{0.7cm}
\noindent{\textbf{keywords}: chatbot, well being, questionnaires, web application}\\

\vspace{0.7cm}
\noindent{\textbf{Abstract}}\\

This project focuses on the development of a conversational agent that allows the collection of data related to the well-being of people. It is part of the POSTCOVID-AI project as a scalable solution to its method of data collection using other technologies. The objective is to facilitate this data collection for subsequent analysis through a tool that is easy to use and attractive to all users.\vspace{0.3cm}

The chatbot guides the user through specific questions about their mood, daily activities, sleeping and eating habits among other aspects. In addition to collecting data, the bot is programmed through a set of established instructions that allow users to interact with it.\vspace{0.3cm}

Thanks to these guidelines, the person can have a conversation with the bot and in turn it will ask a series of previously defined questions. All this can be configured through a web application that can be accessed to modify the questions to be asked, group them, add messages to the bot, plan questionnaires and consult the data in real time.

\chapter*{}
\thispagestyle{empty}

\noindent\rule[-1ex]{\textwidth}{2pt}\\[4.5ex]

Yo, \textbf{Juan del Río Gómez}, alumno de la titulación GRADO EN INGENIERÍA INFORMÁTICA de la \textbf{Escuela Técnica Superior
de Ingenierías Informática y de Telecomunicación de la Universidad de Granada}, con DNI 46069380N, autorizo la ubicación de la siguiente copia de mi Trabajo Fin de Grado en la biblioteca del centro para que pueda ser consultada por las personas que lo deseen.

\vspace{5cm}

\includegraphics[width=0.4\textwidth]{imagenes/firma.png}\\[0.5cm]

\noindent Fdo: Juan del Río Gómez


\vspace{2cm}

\begin{flushright}
Granada a 8 de octubre de 2023 .
\end{flushright}


\chapter*{}
\thispagestyle{empty}

\noindent\rule[-1ex]{\textwidth}{2pt}\\[4.5ex]

D. \textbf{Oresti Baños Legrán(tutor1)}, Profesor del Área de Ingeniería de Sistemas y Automática del Departamento de Ingeniería de Computadores, Automática y Robótica de la Universidad de Granada.

\vspace{0.5cm}

D. \textbf{Miguel Damas Hermoso (tutor2)}, Profesor del Área de de Ingeniería de Sistemas y Automática del Departamento de Ingeniería de Computadores, Automática y Robótica de la Universidad de Granada.


\vspace{0.5cm}

\textbf{Informan:}

\vspace{0.5cm}

Que el presente trabajo, titulado \textit{\textbf{POSTCOVID-AI Chatbot}},
ha sido realizado bajo su supervisión por \textbf{Juan del Río Gómez (alumno)}, y autorizamos la defensa de dicho trabajo ante el tribunal
que corresponda.

\vspace{0.5cm}

Y para que conste, expiden y firman el presente informe en Granada a 8 de Octubre de 2023.

\vspace{1cm}

\textbf{Los directores:}

\vspace{5cm}

\noindent \textbf{Oresti Baños Legrán (tutor1) \ \ \ \ \ Miguel Damas Hermoso(tutor2)}

\chapter*{Agradecimientos}
\thispagestyle{empty}

       \vspace{1cm}

En primer lugar quisiera agradecer a mi familia, sin su ayuda hoy no estaría donde estoy.

A todos mis compañeros de carrera y amigos, siempre dispuestos y atentos a ayudarme a resolver problemas del día a día.

Y a todos los profesores que me han aportado experiencias positivas gracias a su enseñanza durante el transcurso de esta bonita etapa. 
    


